
\section{Problem to solve: Hemoglobin}

The metalloprotein Hemoglobin (\ttt{Hgb}) is the iron-containing protein able to transport oxygen, essential to get energy from aerobic metabolic reactions, through red blood cells of almost every vertebrate. The first atomic structure of \ttt{Hgb} was determined in 1960 by X-ray crystallography \citep{perutz1960}. \ttt{Hgb} was, alongside myoglobin, the first structure solved by this methodology. Due to its emblematic prominence in structural biology History, we have selected \ttt{Hgb} to model its atomic structure.\\

\ttt{Hgb} is a relatively small macromolecule (molecular weight of 64 KDa) that shows C2 symmetry. This heterotetramer is constituted by four globular polypeptide subunits, two $\alpha$ and two $\beta$ monomers with 141 and 146 aminoacids in human \ttt{Hgb}, respectively. Each subunit associates to a prosthetic heme group, that consists in an iron (Fe) ion and the heterocyclic ring of porphyrin. Although the molecule is able of binding oxygen only in the reduced ferrous status, human \ttt{Hgb} is commercially distributed in its nonfunctional oxidized ferric status as \ttt{metHgb}. The atomic structure of the human \ttt{metHgb} specimen was inferred by \citet{khoshouei2017} for the first time from the electron density volume obtained by cryo-EM and using the Volta phase plate. The volume, at 3.2\AA\ resolution, and its atomic interpretation (\ffigure{fig:model_building_example}) are available in the Electron Microscopy Data Bank (\ttt{EMDB}) and Protein Data Bank (\ttt{PDB}) with accession numbers \ttt{EMD-3488} and \ttt{PDB-5NI1}, respectively.\\

This tutorial will guide us in the deduction process of the human \ttt{metHgb} atomic structure using the \scipion framework, the 3D map and the protein sequences as starting input data, as well as reference atomic structures as homologous models. %true data and some fictitious restraints.
